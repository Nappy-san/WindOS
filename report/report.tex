\documentclass[a4paper]{article}
\usepackage[utf8]{inputenc}
\usepackage[T1]{fontenc}
\usepackage{amsmath}
\usepackage{amssymb}
\usepackage{amsthm}
\usepackage{listings}
\usepackage{enumitem}
\usepackage{relsize}
\usepackage{graphicx}
\usepackage[margin=1.0in]{geometry}


\author{Pluvinage Lucas, Fehr Mathieu}
\title{Système et réseaux : Rendu de projet}

\begin{document}

\maketitle

\begin{abstract}
  Pour ce projet, nous nous sommes intéressés à l'écriture d'un noyau en C pour
  Raspberry Pi (1 et 2). L'objectif était à la base de pouvoir accéder à l'OS en
  réseau, ce qui n'a pas été implémenté au final, faute de temps. Le noyau
  implémenté dispose cependant de multiples fonctionnalités similaires à un noyau
  UNIX. Ainsi nous disposons d'un système d'exploitation préemptif, avec notamment
  un mode utilisateur, un système de mémoire virtuelle et un système de fichier.
\end{abstract}

\tableofcontents
\newpage

\section{Téléchargement et compilation du projet}
\subsection{Sources et documentation}

Les sources du projet sont clonables via github, à l'addresse https://github.com/Nappy-san/WindOS.git.\\

Une documentation Doxygen a été compilé avec le projet, pour le rendre plus lisible.
% TODO  ajouter la documentation dans le makefile ou dans un dossier
% TODO finir subsection

\subsection{Modules utilisés et dépendances}

Le dépôt contient un sous-module:
\begin{itemize}
\item qemu-fvm, qui est un fork de qemu permettant d'émuler le raspberry pi 2.
\end{itemize}
% TODO voir si
% demander de faire un git submodule init/update
% ou de le mettre dans le makefile
\ \\
D'autres dépendances sont nécessaire pour pouvoir faire tourner l'OS:
\begin{itemize}
\item genext2fs, qui permet de créer un système de fichier qui sera chargé en RAM.
\item minicom, qui permet de lire le port série.
\item arm-none-eabi-gcc, qui est le compilateur croisé nous permettant de compiler pour
l'architecture ARM.
\end{itemize}

\subsection{Compilation du projet}

% TODO Ajouter un make init pour initialiser les submodules et copier les
% fichiers pour USPi
% TODO voir readme
%TODO changer make all en gras type commande
Pour compiler le projet, il faut lancer ``make all'', qui compilera le kernel,
et les programmes utilisateurs. Ensuite, pour lancer le kernel sur qemu, il
suffit d'executer ``make runs''.\\

Pour lancer le kernel sur raspberry pi 2 (le
kernel ne marchant pas sur raspberry pi 1), il faut copier les fichiers contenus
dans le dossier ``img'', et les copier dans la première partition fat de la
carte SD inséré dans le raspberry pi. Pour pouvoir afficher dans un terminal la
sortie du raspberry pi, il faut brancher sur les pins connectés au controller
uart un cable USBTTL, puis lancer minicom sur un terminal (avec la commande
``sudo minicom -c on -b 115200 -D dev/ttyUSB0'' si la sortie est située dans
ttyUSB0).


\section{Modules du kernel}
Le kernel contient plusieurs modules intéressants, comme la gestion du port série
pour envoyer des données à un ordinateur, l'utilisation du timer ARM,
l'utilisation du MMU pour la gestion de la mémoire virtuelle, une abstraction de
système de fichier (VFS) sur lequel deux systèmes de fichiers sont montés, et
une mailbox permettant de communiquer avec le VideoCore.

\subsection{Serial}
Le Raspberry Pi dispose de plusieurs UART (Universal Asynchronous Receiver Transmitter).
Nous utilisons le Mini-UART qui est une version simplifiée implémentant UART avec
moins d'options à contrôler. Avec un baudrate fixé à 115200 bps, cela nous suffit
largement pour le développement d'un terminal.
Le mini-UART dispose d'un buffer de lecture et d'écriture de 8 byte, ce qui permet
de ne pas avoir à suivre précisément les interruptions levées par ce dernier:
on peut se contenter de récupérer les données reçues à chaque tick de timer.

\subsection{Timer}
Le timer est très utile pour l'ordonnancement des processus, puisqu'il permet de
lancer des interruptions à une fréquence donnée. Nous utilisons le timer de la
carte BCM2835/2836. Nous l'avons implémenté à partir de la documentation du
BCM2835 ARM Peripherals, à la page 196.\\

Le timer permet de retarder l'exécution d'une fonction, et surtout de générer
les interruptions de contrôle indispensables dans le cadre d'un OS préemptif.

\subsection{Mémoire virtuelle}

Le MMU (Memory Management Unit, décrit dans la partie B4 de l'arm ARM
(https://www.scss.tcd.ie/~waldroj/3d1/arm\_arm.pdf)) est utilisé pour permettre
de rediriger des pages virtuelles vers des pages physiques.\\

Pour cela, nous avons implémenté dans mmu.c des fonctions bas niveau permettant de
rediriger des pages de toute tailles via les translation table base (dont le
système est très bien expliqué dans l'arm ARM partie B4.7).\\

Ces fonctions sont ensuites utilisées par memalloc.c pour avoir une gestion de
la mémoire virtuelle, en retenant les pages physiques déjà allouées.

\subsection{Système de fichiers}
Après une tentative infructueuse d'implémenter FAT, c'est EXT2 qui a été utilisé
pour représenter les fichiers en mémoire.
Au lieu de brancher directement Ext2 aux appels systèmes et autres fonctionnalités
noyau, une interface a été développée afin de faire abstraction entre le driver
et le reste du kernel. \\
Ainsi chaque système de fichiers développé implemente des fonctions définies
par le VFS (Virtual File System) ce qui permet au VFS de traiter indifféremment
chaque système de fichier. Cela permet en plus de pouvoir monter un système de
fichier sur un autre, et ainsi accéder à plusieurs systèmes fichiers à partir de
l'arborescence. \\
Deux drivers de systèmes de fichiers ont été implémentés:
\begin{itemize}
	\item EXT2: Ce système de fichiers, utilisé dans l'écosystème Unix avant
	d'être remplacé par EXT3/4, utilise une structure à base d'i-node et de blocks.
	\item DEV: Pour représenter les différentes interfaces disponibles dans l'OS,
	(inspiré de /dev chez Unix), un pseudo-système de fichier a été implémenté.
	Il permet par exemple d'accéder au port série (/dev/serial).
\end{itemize}

\subsection{Mailbox}

La mailbox est ce qui permet la communication entre le VideoCore (le GPU) et le
processeur. Cela permet au processeur de demander des informations au GPU, ou
de demander au GPU d'effectuer des actions, comme alimenter en énergie un
périphérique. La mailbox est bien expliqué sur la page
https://github.com/raspberrypi/firmware/wiki/Mailboxes, mais reste très mal
documentée sinon.\\

Pour communiquer avec la mailbox, il faut lui envoyer un pointeur sur l'
instruction demandée, qui est une structure contenant le tag de
l'instruction, ainsi que l'espace mémoire nécessaire pour stocker les données de
retour ou d'entrée. Il est possible d'envoyer jusqu'a 8 d'instructions en même
temps, mais pour faciliter l'implémentation, et puisque nous utilisons peu la
mailbox, nous n'envoyons qu'une instruction à la fois.

\section{Gestion de l'User mode}
L'user mode est correctement géré par notre noyau, sous un modèle préemptif.
Un timer déclenche des interruptions à intervalle régulier, permettant au noyau
de reprendre la main sur les processus et d'exécuter l'ordonnanceur.

\subsection{Gestion de la mémoire virtuelle}
La mémoire virtuelle permet d'isoler les processus les uns des autres, ainsi que
de leur faire croire qu'ils ont accès à toute la RAM. Nous avons placé le kernel
dans la partie haute de la mémoire virtuelle (de 0x80000000 à 0xFFFFFFFF), pour
y avoir accès directement lors des interruptions, sans avoir à changer la tables
de translations. Grace aux fonctions définies dans memalloc, le kernel peut
récupérer facilement les pages physiques libres disponibles, pour les affecter
aux processus demandant de la mémoire. \\
Le mapping choisi est le suivant:
\begin{itemize}
	\item 0x00000000 - 0x80000000: Espace utilisateur.
	\item 0x80000000 - 0xbf000000: Mapping linéaire vers la mémoire physique.
	\item 0xbf000000 - 0xc0000000: Mapping vers les addresses de périphériques.
	\item 0xc0000000 - 0xf0000000: Tas (heap) du noyau alloué dynamiquement.
	\item 0xf0000000 - 0xffffffff: Code et segment de données noyau.
\end{itemize}
\subsection{Bibliothèque standard}
Le système disposant d'un MMU, les programmes sont compilables avec un nombre
minimal d'options, à condition d'utiliser arm-none-eabi-gcc. Cette toolchain
est compilée avec Newlib, une bibliothèque de développement bare metal
implémentant en partie la libc et demandant un nombre minimal d'appels systèmes. \\
Ainsi pour communiquer avec le noyau, les programmes disposent d'un certain
nombre d'appels systèmes Linux, décrits dans include/syscalls.h.
Des fonctionnalités de terminal furent aussi développées, elles sont décrites
dans include/termfeatures.h.

\subsection{Ordonnancement des processus}

L'ordonnancement des processus se fait de manière assez simple. On fixe le
nombre maximum de processus pouvant être éxecutés en parallèle. Les ID que l'on
donne aux processus existants correspond à leur addresse dans un tableau
contenant tout les processus (existants ou ``libres''). On a de plus un tableau
listant tout les id libres, et un contenant tout les id correspondant à des
processus actifs (et qui de plus ne sont pas dans un wait), et un tableau
contenant les processus zombies.

Pour récupérer les processus à executer, il suffit donc de parcourir la liste
des processus actif de gauche à droite, puis recommencer.

\subsection{Appels systèmes}
L'objectif initial était de faire un système s'apparentant à un noyau Linux.
Ainsi les appels systèmes développés suivent les mêmes spécifications.
L'encodage des erreurs se fait via la variable errno, que l'on transmet au
processus en renvoyant -ernno, une valeur négative étant signe d'erreur, ce qui
est décodé dans la bibliothèque utilisateur.

\section{Séquence de boot}

La séquence de boot d'un Raspberry Pi n'est pas la même que pour un ordinateur
classique. C'est le GPU qui démarre le CPU, et qui facilite la tâche du boot.

\subsection{Lancement du code via le GPU}

Pour booter le Raspberry Pi, le GPU cherche, dans la première partition FAT de la
carte SD, un fichier nommé bootcode.bin (firmware fourni par l'organisation Raspberry Pi),
qui va lui-même lire le fichier start.elf, qui va s'occuper de lire le fichier CONFIG.TXT, pour
ensuite trouver le fichier image à exécuter, qui est ``kernel.img'' dans notre
cas. %TODO kernel.img mieux formatter
C'est donc l'image kernel.img que nous devons compiler pour le noyau. La
première instruction qui va être executée est à l'addresse 0x8000 pour le
raspberry pi, et 0x10000 pour QEMU.

\subsection{Mise en place des interruptions et du MMU}

La première chose que nous faisons lors du boot est de vérifier que nous somme
bien en mode superviseur (car le raspberry pi 2 démarre en hyperviseur, alors
que QEMU démarre en superviseur). Ensuite, nous enregistrons la table
d'interruptions à l'addresse 0, puis nous initialisons les stacks pour les
différents modes, puis le MMU. Nous laissons la première page virtuelle pointer
sur la première page physique, afin de pouvoir sauter vers le début du code C du
kernel, situé dans la moitié haute de la RAM.

\subsection{Initialisation des modules et du mode utilisateur}

Une fois dans le code C, la première étape est d'initialiser les modules (comme
le timer, ou les interruptions, ou encore le système de fichiers). Ensuite, on
s'occupe de charger en mémoire le programme init. On lance alors le mode
utilisateur, en démarrant dans ce programme.

\section{Programmes fournis avec l'OS}
\begin{itemize}
	\item cat: concatène les fichiers passés en paramètres et affiche le résultat sur stdout.
	\item cp: copie un fichier ou un dossier vers une destination.
	\item init: initialise le premier shell, et récupère les processus zombies.
	\item ls: liste les entrées d'un dossier. (-i pour afficher les inodes, -a pour afficher les dossiers cachés, -l pour afficher en ligne)
	\item mkdir: crée un dossier dans le répertoire courant.
	\item pico: un éditeur de fichiers minimaliste.
	\item pwd: affiche le dossier courant.
	\item rm: supprime un fichier ou un dossier en mode récursif (-r)
	\item show: affiche un fichier au format bitmap (base résolution de préférence).
	\item touch: créer un fichier.
	\item wesh: WindOS Experimental Shell, l'invite de commande lancé au démarrage de l'OS.
\end{itemize}

\section{Difficultés rencontrées}
On plante sur une instruction, 3 instruction avant on a r0=r5, mais quand ça
plante on a plus r0=r5
Port série qui dit nope parfois
fp wtf parfois
hypervisor mode lol bug


\end{document}
